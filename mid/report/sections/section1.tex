\section{Giới thiệu tổng quan}

Advance Persistent Threats là một quá trình tấn công mạng tinh vi trong thời gian dài, kẻ tấn công không bị phát hiện trong một thời gian dài và do đó, tiếp cận, đánh cắp, phá hoại dữ liệu quan trọng trong hệ thống. Nó sử dụng kết hợp nhiều công cụ và kỹ thuật phức tạp để xâm nhập và duy trì quyền truy cập tới hệ thống. Kẻ tấn công có mục tiêu dài hạn cụ thể (như gián điệp) và liên tục theo dõi và duy trì tương tác với hệ thống. Kẻ tấn công thường là tổ chức, có kỹ thuật và tổ chức tốt, được tài trợ thường xuyên, đôi khi do nhà nước tài trợ. Tóm lại, Advanced Persistent Threats (APTs) là những cuộc tấn công mạng kéo dài, tinh vi, có chủ đích rõ ràng và do những tổ chức lành nghề được tài trợ thực hiện \cite{alshamrani2019survey}. Hầu hết các cuộc tấn công APT đều liên quan đến lỗ hổng zero-day và rất khó phát hiện. 

Các nỗ lực nhằm phát hiện APT chủ yếu dựa vào các phương pháp: (1) xây dựng rules-base dựa trên các các mẫu APT phổ biến và so khớp với audit logs, (2) sử dụng thống kê các thành phần trong hệ thống: system entities, tương tác mạng... để phát hiện bất thường, (3) sử dụng các kỹ thuật học sâu để mô hình hóa tấn công APT hoặc hành vi hệ thống, sau đó phát hiện APT bằng cách phân loại hoặc phát hiện bất thường.

Các phương pháp trên đã chứng minh được sự hiệu quả đối với APT, nhưng còn một số thách thức: (1) học giám sát đòi hỏi phải có dữ liệu lớn về APT để train model, khi gặp APT mới thì khó phát hiện đây là tấn công, (2) các phương pháp thống kê không trích xuất được mối tương quan phức tạp trong log của hệ thống, (3) các phương pháp sử dụng DL có ưu điểm nhiều, nhưng yêu cầu quá nhiều tài nguyên tính toán, dẫn tới khó áp dụng trong thực tế.